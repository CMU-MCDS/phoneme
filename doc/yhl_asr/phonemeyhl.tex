\documentclass{article}

% if you need to pass options to natbib, use, e.g.:
% \PassOptionsToPackage{numbers, compress}{natbib}
% before loading nips_2017
%
% to avoid loading the natbib package, add option nonatbib:
% \usepackage[nonatbib]{nips_2017}

%\usepackage{nips_2017}

% to compile a camera-ready version, add the [final] option, e.g.:
\usepackage[final]{nips_2017}

\usepackage[utf8]{inputenc} % allow utf-8 input
\usepackage[T1]{fontenc}    % use 8-bit T1 fonts
\usepackage{hyperref}       % hyperlinks
\usepackage{url}            % simple URL typesetting
\usepackage{booktabs}       % professional-quality tables
\usepackage{amsfonts}       % blackboard math symbols
\usepackage{nicefrac}       % compact symbols for 1/2, etc.
\usepackage{microtype}      % microtypography

\usepackage{amsmath}	% for \begin{align}
\usepackage{graphicx}	% for \includegraphics{filename}
\usepackage{subcaption}	% for \begin{subfigure}[t]{0.5\textwidth}
\usepackage{enumitem}	% for using (a), (b), ...
\usepackage{courier}	% for \texttt{}

\newcommand{\bb}[1]{\boldsymbol{#1}}


\title{Progress Update}

% The \author macro works with any number of authors. There are two
% commands used to separate the names and addresses of multiple
% authors: \And and \AND.
%
% Using \And between authors leaves it to LaTeX to determine where to
% break the lines. Using \AND forces a line break at that point. So,
% if LaTeX puts 3 of 4 authors names on the first line, and the last
% on the second line, try using \AND instead of \And before the third
% author name.

\author{
	Yu-Hsiang Lin
		%\thanks{Use footnote for providing further information about author (webpage, alternative address)---\emph{not} for acknowledging funding agencies.}%
		\\
	Language Technologies Institute\\
	Carnegie Mellon University\\
	%\texttt{\{chianyuc, jeanl1, ziruil, yuhsianl, yuyanz1\}@andrew.cmu.edu} \\
	\texttt{yuhsianl@andrew.cmu.edu} \\
	%% examples of more authors
%	\And
%	Graham Neubig \\
%	Language Technologies Institute\\
%	Carnegie Mellon University\\
%	\texttt{gneubig@cs.cmu.edu} \\
  %% \AND
  %% Coauthor \\
  %% Affiliation \\
  %% Address \\
  %% \texttt{email} \\
  %% \And
  %% Coauthor \\
  %% Affiliation \\
  %% Address \\
  %% \texttt{email} \\
  %% \And
  %% Coauthor \\
  %% Affiliation \\
  %% Address \\
  %% \texttt{email} \\
}

\begin{document}
% \nipsfinalcopy is no longer used

\maketitle



% ---------------------------------------------------------
% ---------------------------------------------------------

\begin{abstract}

Place holder \cite{Adams2017}.

\end{abstract}

% ---------------------------------------------------------
% ---------------------------------------------------------

% ---------------------------------------------------------

\section{Question to answer}

How can we create a speech recognition system that transcribes low-resource languages into International Phonetic Alphabets (IPAs) as accurately and quickly as possible?



% ---------------------------------------------------------

\section{Naive way to proceed}

Based on \cite{Neubig2018}.

\begin{enumerate}

\item Prepare data: $N$ high-resource languages that have transcription from speech to IPAs (or speech $\rightarrow$ language $\rightarrow$ IPA?), and a target low-resource language with limited transcription.
	
\item Train a multi-lingual ASR ($N$ high-resource languages $\rightarrow$ IPA) model: CTC objective function $+$ biLSTM.

(Model parameters $=$ LSTM only? Input acoustic representation is engineered. Output phonemic sequence is computed by CTC decoding.)

Can be either warm-start (training data contains target low-resource languages) or cold-start (training data contains only high-resource languages).

\item Given a target low-resource language, incrementally train (fine-tune) the multi-lingual model: 3 possibilities $\Rightarrow$

	\begin{enumerate}
	
	\item Train on only the target low-resource language.
	
	\item Select a high-resource language (that is used previously in the multi-lingual model training) that is similar to the target low-resource language, and train on their data combined (concatenation for accuracy, or sampling for speed).
	
	\item Think of some way to use the untranscribed speech recording (could actually be plenty of them) to help the training on the target low-resource language.
	
	(May be an independent topic outside the multi-lingual setting?)
	
	\end{enumerate}

\end{enumerate}



% ---------------------------------------------------------

\section{Other possibilities}

\begin{itemize}

\item Meta learning \cite{Gu2018}: learn a good initialization.

\item Some way to use the untranscribed speech recording to help the training on the target low-resource language (without multi-lingual pre-trained model).

Will the (plenty of) untranscribed data help regularization/generalization?

$\Rightarrow$ An idea: The typical pattern of semi-supervised learning is (similar to EM): Do iterations. Within each iteration we have two stages. First stage: transcribe the untranscribed data using the current model. Second stage: either maximize the expectation of the model of predicting such transcription (EM), or compute the similarity between the unlabeled speech and the labeled speech with the same predicted labels, and penalize the diversity.

Question is: how to update the model (at least in the second way)? How can I use the signal coming from the diversity to update my model?

\item Cotraining? Separate features into two groups and train two ASR models?

\end{itemize}



% ---------------------------------------------------------
% ---------------------------------------------------------

%\subsubsection*{Acknowledgments}

% Use unnumbered third level headings for the acknowledgments.



% ----------------------------------------------------
% ----------------------------------------------------

\bibliographystyle{plain}
\bibliography{phonemebib}

\end{document}
