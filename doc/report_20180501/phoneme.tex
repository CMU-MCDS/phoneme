\documentclass{article}

% if you need to pass options to natbib, use, e.g.:
% \PassOptionsToPackage{numbers, compress}{natbib}
% before loading nips_2017
%
% to avoid loading the natbib package, add option nonatbib:
% \usepackage[nonatbib]{nips_2017}

%\usepackage{nips_2017}

% to compile a camera-ready version, add the [final] option, e.g.:
\usepackage[final]{nips_2017}

\usepackage[utf8]{inputenc} % allow utf-8 input
\usepackage[T1]{fontenc}    % use 8-bit T1 fonts
\usepackage{hyperref}       % hyperlinks
\usepackage{url}            % simple URL typesetting
\usepackage{booktabs}       % professional-quality tables
\usepackage{amsfonts}       % blackboard math symbols
\usepackage{nicefrac}       % compact symbols for 1/2, etc.
\usepackage{microtype}      % microtypography

\usepackage{amsmath}	% for \begin{align}
\usepackage{graphicx}	% for \includegraphics{filename}
\usepackage{subcaption}	% for \begin{subfigure}[t]{0.5\textwidth}
\usepackage{enumitem}	% for using (a), (b), ...
\usepackage{courier}	% for \texttt{}

\newcommand{\bb}[1]{\boldsymbol{#1}}


\title{Intelligent Linguistic Annotation Interface}

% The \author macro works with any number of authors. There are two
% commands used to separate the names and addresses of multiple
% authors: \And and \AND.
%
% Using \And between authors leaves it to LaTeX to determine where to
% break the lines. Using \AND forces a line break at that point. So,
% if LaTeX puts 3 of 4 authors names on the first line, and the last
% on the second line, try using \AND instead of \And before the third
% author name.

\author{
	Chian-Yu Chen, Jean Lee, Zirui Li, Yu-Hsiang Lin, Yuyan Zhang
		%\thanks{Use footnote for providing further information about author (webpage, alternative address)---\emph{not} for acknowledging funding agencies.}%
		\\
	Language Technologies Institute\\
	Carnegie Mellon University\\
	\texttt{\{chianyuc, jeanl1, ziruil, yuhsianl, yuyanz1\}@andrew.cmu.edu} \\
	%% examples of more authors
	\And
	Graham Neubig \\
	Language Technologies Institute\\
	Carnegie Mellon University\\
	\texttt{gneubig@cs.cmu.edu} \\
  %% \AND
  %% Coauthor \\
  %% Affiliation \\
  %% Address \\
  %% \texttt{email} \\
  %% \And
  %% Coauthor \\
  %% Affiliation \\
  %% Address \\
  %% \texttt{email} \\
  %% \And
  %% Coauthor \\
  %% Affiliation \\
  %% Address \\
  %% \texttt{email} \\
}

\begin{document}
% \nipsfinalcopy is no longer used

\maketitle



% ---------------------------------------------------------
% ---------------------------------------------------------

\begin{abstract}
	Based on \cite{Adams2017}.
\end{abstract}

% ---------------------------------------------------------
% ---------------------------------------------------------

% ---------------------------------------------------------

\section{Data: Griko--Italian}

\subsection{Data acuqisition}

	The human process of gathering and producing data is roughly as follows:
		\begin{enumerate}
			\item Griko speech $\rightarrow$ Griko transcription
			\item Griko transcription $\rightarrow$ Italian glossing
			\item Produced POS and morphosyntactic tags for each Griko token, and everything follows
		\end{enumerate}

	Logically, the data can be understood by the following relations:
		\begin{enumerate}
			\item Raw wav files (Griko speech)
			\item wav2gr (written Griko token) $\Leftarrow$ extracted from raw wav files
			\item silences $\Leftarrow$ extracted from raw wav files%
				\footnote{But silence does not exactly fill the spaces between the intervals listed in wav2gr.}
			\item Transcriptions = concatenating wav2gr
			\item itgloss (Italian gloss, an Italian corresponding word is given to each Griko token) $\Leftarrow$ obtained from wav2gr
			\item Translations $\Leftarrow$ performed from raw wav files (not necessary)
			\item wav2it $\Leftarrow$ some Italian words in translations are aligned with corresponding segments of raw wav files
		\end{enumerate}
		For the transcription task, the most important data are the first 4 items.



% ---------------------------------------------------------

\subsection{Basic statistics}

	Total 331 data in this dataset, from number 1 to 332, with number 5 missing.



% ---------------------------------------------------------

\subsection{Raw wav file format}

	Python document for wave library: https://docs.python.org/3/library/wave.html

	Useful ref: https://stackoverflow.com/questions/18625085/how-to-plot-a-wav-file


% ---------------------------------------------------------

\subsection{wav2gr format}

	Griko token, start time, end time. Time in the unit of 10 ms.



% ---------------------------------------------------------

\subsection{Silences format}

	Start time, end time. Time in the unit of 10 ms.



% ---------------------------------------------------------

\subsection{Transcriptions format}

	The string of Griko transcription.



% ---------------------------------------------------------

\subsection{Example: data number 6, clean, only one speaker}
	
	Raw wav file: [raw/6.wav]
	
	Griko tokens: [wav2gr/6.words] \\			
		ti 57 70 \\
		k\`{a}nni 70 93 \\
		e 93 102 \\
		Anna 102 156 \\
		o 185 198 \\
		s\`{a}mba 198 240 \\
		porn\`{o} 255 306

	Silences: [silences/6.txt] \\
		0 54 \\
		155 185 \\
		239 255 \\
		307 349

	Transcription: [transcriptions/6.gr] \\
		ti k\`{a}nni e Anna o s\`{a}mba porn\`{o}



% ---------------------------------------------------------

\subsection{Example: data number 10, has another background speaker}
	
	Raw wav file: [raw/10.wav]
	
	Griko tokens: [wav2gr/10.words] \\
		all\`{o}ra 81 120 \\
		f\`{e}ti 120 149 \\
		p 149 156 \\
		\`{o}rkete 156 225 \\
		\`{e}rkome 249 300 \\
		map\`{a}le 293 351 \\
		ett\`{u} 351 390 \\
		ce 410 428 \\
		tr\`{o}o 428 453 \\
		podd\`{u} 453 497 \\
		pasticci\`{o}ttu 497 584

	Silences: [silences/10.txt] \\
		0 80 \\
		224 249 \\
		391 410 \\
		589 599

	Transcription: [transcriptions/10.gr] \\
		all\`{o}ra f\`{e}ti p \`{o}rkete \`{e}rkome map\`{a}le ett\`{u} ce tr\`{o}o podd\`{u} pasticci\`{o}ttu



% ---------------------------------------------------------

\subsection{Example: data number 30, two equally strong speakers}
	
	The female speaker is transcribed.
	
	Raw wav file: [raw/30.wav]
	
	Griko tokens: [wav2gr/30.words] \\
		ste 33 49 \\
		kamm\`{e}ni 49 108 \\
		sto' 108 128 \\
		giard\`{i}no 128 197 \\
		sto' 197 219 \\
		c\`{i}po 219 274

	Silences: [silences/30.txt] \\
		0 32 \\
		274 299

	Transcription: [transcriptions/30.gr] \\
		ste kamm\`{e}ni sto' giard\`{i}no sto' c\`{i}po




% ---------------------------------------------------------
% ---------------------------------------------------------

%\subsubsection*{Acknowledgments}

% Use unnumbered third level headings for the acknowledgments.



% ----------------------------------------------------
% ----------------------------------------------------

\bibliographystyle{plain}
\bibliography{phonemebib}

\end{document}
